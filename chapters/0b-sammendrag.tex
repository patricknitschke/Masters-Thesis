\chapter*{Sammendrag}

Autonom navigering i stadig mer komplekse domener byr på nye utfordringer og stiller spørsmål ved effektiviteten og kapasiteten til tradisjonelle modellbaserte metoder. Selv om tradisjonelle metoder har vært vellykkede for ustrukturerte miljøer i det siste, kan usikre, sensor-degraderte eller dynamiske miljøer ikke modelleres og dermed løses med disse metodene. I stedet har læringsbaserte metoder blitt stadig mer populære på grunn av deres evne til å lære kompleks atferd uten eksplisitt programmering, der flere komponenter kan kombineres til en enkelt modell for å takle persepsjons-, prediksjons- og bevegelsesoppgaven til autonom navigering.

I dette temaet utforsker denne oppgaven bruken av forsterkende læring for autonom navigering av en drone gjennom hinderfylt miljøer, med kun et dybdekamera. Vi foreslår en todelt dyp nevrale nettverksmodell som består av en koder-CNN og MLP, der CNN fungerer som persepsjonsmodulen mens MLP er den optimale kontrolleren. Med dette rammeverket mottar modellen vår en dronetilstand og et dybdebilde som input og kartlegger dette til en hastighets- og girhastighetsreferanse for å nå et spesifisert mål i tre dimensjoner.

For å løse oppgaven presenterer vi problemet som en uovervåket representasjonslærings- og forsterkende læringsoppgave. CNN er opplært som en koder for VAE som lærer å rekonstruere dybdebilder, mens MLP lærer å bruke VAE latent kode som en dybderepresentasjon av miljøet, for å kunne navigere i miljøet. Vi introduserer en tilpasset rekonstruksjonsfeil for VAE for å spesifisere kollisjonsspesifikke funksjoner som bør prioriteres i dybdekodingen. Vi introduserer også en ny belønningsfunksjon for forsterkende læringsmiddel som motiverer både veipunktnavigasjon og kollisjonsunngåelse.

Ved ytterligere å bruke storskala parallellisme, presenterer vi opplæringen og evalueringen av vår endelige forsterkende læringspolicy, som oppnår en suksessrate på 92,5\% i gjennomsnitt over fire kjente miljøer på 20$\ ganger $10 med ulik grad av rot. Agenten viser god robusthet når en Gaussisk multiplikativ støy $\epsilon_n \sim \mathcal{N}(1, 0,2)$ brukes på alle tilstander og handlinger, med en suksessrate på 87,5\% på tvers av de fire miljøene. Imidlertid identifiserer vi noen begrensninger med modellen vår -- nemlig avhengighet av nøyaktige dybderepresentasjoner og en dårlig generalisering til større miljøer. Til slutt, som videre arbeid, bør vi trene modulene våre til å håndtere støyende dybdebilder, legge til modifikasjoner for å ta hensyn til generalisering, og legge til en prediksjonsmodul i form av en LSTM eller transformator for å forbedre ytelsen ytterligere.